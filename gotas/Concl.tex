\section{Conclusão}
À partir do experimento realizado foi possível medir a carga elétrica transportada pelas gotas em um campo elétrico conhecido. Analisando os valores obtidos foi possível verificar a quantização da carga com a obtenção do valor $e=1.12\times 10^{-19}C$ para a carga do eletron que é bastante próximo do valor real, $e=1.6021765 \times 10^{-19}C$.
Atribui-se o desvio percentual de 29.9\% aos erros associados ao reflexo humano entre verificar uma ação ocorrendo e medir seu tempo. Além disso, há também os erros associados aos aparelhos de medida utilizados.
Apesar de ser uma experimento relativamente fácil, este exije muita paciência e atenção devido ao tamanho microscópico das gotas e a dificuldade de acompanhar uma única gosta em meio a várias.
