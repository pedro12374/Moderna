\section{Resumo}
A experiência da gota de óleo foi um experimento conduzido pelo físico Robert Andrews Millikan com o intuito de medir a carga elétrica do elétron. A medição feita por Millikan, que foi reproduzida no experimento descrito nesse relatório, consistiu em balancear as forças elétrica e gravitacional em minúsculas gotas de óleo carregadas e suspensas entre duas placas metálicas eletrizadas. O valor da carga do elétron pode ser obtida ao se saber o valor do campo elétrico gerado entre a placas. 
O aparato utilizado consiste em duas placas metálicas eletrizadas onde as gotas carregadas ficam em suspensão, entrando em movimento mediante variação mo campo elétrico. O movimento das gotas é visto através de um pequeno microscópio acoplado a um orifício no espaço em que as partículas se encontram.
\section{Introdução}
Os gregos foram os primeiros a relatar a existência da eletricidade ao estudar o efeito do atrito sobre âmbar e seda, entratanto não surgiram teorias explicando esse fenômeno até o início do século XVIII, quando Bemjamin Franklin propôs que existia um fluido elétrico em determinadas formas de matéria. O excesso fo fluído produziria carga positiva no corpo enquanto a falta produziria carga negativa. Franklin ainda propôs a existência de uma partícula elétrica pequena o suficiente para que pudesse permear a matéria. Os experimentos de Faraday na área de eletrólise - que mostraram que quando uma corrente elétrica passa por um eletrólito parte da massa dos compostos depositados em cada eletrodo é proporcional à massa atômica dos compostos - sustentaram a teoria de Franklin.

Até o início do século XIX não se sabia a carga do elétron, apenas a razão carga-massa $(C/m)$ que foi obtida por J. J. Thomson no fim do século anterior. Numa tentativa mais bem sucedida, Robert Andrews Millikan desenvolveu o "experimento da gota de óleo" para tentar obter o valor da carga do elétron. O experimento consistia basicamente em borrifar gotículas de óleo dentro em um espaço fechado entre duas placas. Nessas placas, as gotículas se eletrizavam por atrito, e eram retro-iluminadas, o que as permitia serem vistas com um aparelho ótico pelo espalhamento da luz. Carregando e descarregando as placas, Millikan, por meio de uma comparação entre as reações das gotas nos dois momentos, obteve valores múltiplos inteiros, e pequenos, de $1,59\times 10^{-19}C$, valor que se aproxima muito do valor utilizado atualmente $(1,602\times 10^{-19}C)$, que seria a carga elementar do eletron ($e$).
