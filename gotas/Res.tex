\section{Resumo}
A experiência da gota de óleo foi um experimento conduzido pelo físico Robert Andrews Millikan com o intuito de medir a carga elétrica do elétron. A medição feita por Millikan, que foi reproduzida no experimento descrito nesse relatório, consistiu em balancear as forças elétrica e gravitacional em minúsculas gotas de óleo carregadas e suspensas entre duas placas metálicas eletrizadas. O valor da carga do elétron pode ser obtida ao se saber o valor do campo elétrico gerado entre a placas. 
O aparato utilizado consiste em duas placas metálicas eletrizadas onde as gotas carregadas ficam em suspensão, entrando em movimento mediante variação mo campo elétrico. O movimento das gotas é visto através de um pequeno microscópio acoplado a um orifício no espaço em que as partículas se encontram.
\section{Introdução}
No fim do século XIX surgiram as primeiras estimativas da grandeza da razão e/m com o físico Pieter Zeeman, inspirado pelos trabalhos de Faraday o neerlandês obteve provas de partículas atômicas com relação definida entre carga e massa. Zeeman, por meio de uma análise espectral da luz emitida pelos átomos na presença de um campo magnético, concluiu que as partículas que emitiam luz nos átomos possuíam carga negativa, também encontrou um valor para e/m de $1,6\times10^{11} C/kg$, valor muito próximo ao das medições atuais ($1,7596\times10^{11}C/kg$).

mson, com um experimento muito parecido com o que foi reproduzido para este relatório, obteve por meio de medidas diretas valores para a razão carga-massa, utilizando um tubo de raios catódicos, uma voltagem especifica, e um campo magnético perpendicular ao campo elétrico gerado pela tenção aplicada. Thomson media a deflexão que o campo magnético gerava, por meio deste método o cientista não obteve um valor muito satisfatório ($0.7\times10^{11} C/kg$) – apesar de já ter obtido valores mais próximos ao real em diferentes experimentos com a mesma finalidade - devido ao fato de não ter levado em consideração o campo magnético da terra que podia influenciar o raio magnético experimental após a interação com o campo criado por ele. 
