\section{Desenvolvimento Teórico}
\subsection{Razão Carga Massa}
A força magnética atuante em uma partícula eletricamente carregada de carga {\it q} num campo magnético {\it B} é dado pela equação
\begin{equation}
F_m = qv \times B
\end{equation}
Onde {\it v} é a velocidade da partícula. Para o caso em que a velocidade é perpendicular à direção do campo, a equação pode ser simplificada para a forma escalar
\begin{equation}
	F_m = evB
\end{equation}
Em que {\it e} é a carga elementar do elétron. Como os elétrons do feixe realizarão um movimento circular dentro do bulbo de vidro, estes estarão sujeitos a uma força centrípeta de forma
\begin{equation}
	F_c = \frac{mv^2}{r}
\end{equation}
Onde {\it m} é a massa do elétron,{\it v} sua velocidade e {\it r} o raio do movimento circular. Como a força centrípeta é a única força externa agindo sobre o elétron, é possível igualar as duas equações de modo que
\begin{equation}
	F_m = F_c
\end{equation}
\begin{equation}
	evB = \frac{mv^2}{r}
\end{equation}
Como o objetivo é determinar a relação carga/massa, deve-se isolar esse quociente de modo a se obter seu valor em função dos demais valores
\begin{equation}
	\frac{e}{m}=\frac{v}{rB}
	\label{eq: em}
\end{equation}
A velocidade do elétron é determinada a partir da energia cinética dos elétrons sujeitos ao campo magnético, ou seja
\begin{equation}
	eV = \frac{1}{2}mv^2
\end{equation}
\begin{equation}
	\\ v = \left(\frac{2ev}{m}\right)^{\frac{1}{2}}
	\label{eq: v}
\end{equation}
O campo magnético produzido por um par de bobinas de Helmholtz é, nas proximidades do centro dado dado pela equação
 \begin{equation}
 	B = \frac{[N\mu _0]I}{a\left(\frac{5}{4}\right)^{\frac{3}{2}}}
	\label{B}
 \end{equation}
Substituindo \ref{eq: v} e \ref{B} na equação \ref{eq: em},
\begin{equation}
	\frac{e}{m}=\frac{v}{rB}=\frac{2V\left(\frac{5}{4}\right)^3a^2}{[N\mu _0 Ir]^2}
	\label{e/m}
\end{equation}
Onde $V$ é a energia potencial dos elétrons, $a$ o raio das bobinas de Helmholtz, $N$ o número de espiras em cada bobina de Helmholtz, $\mu _0$ a permeabilidade elétrica do meio, $I$ a corrente elétrica gerada nas bobinas e $r$ o raio de feixe de elétrons.

É possível determinar a relação carga/massa facilmente por este último resultado visto que é composto por constantes($N=130$ e $\mu _0 =4\pi 10^{-7}$) e valores que são ajustados nas fontes no decorrer do experimento.
\subsection{Desvios}
  \subsubsection{Desvio de Medidas Diretas}
  Para este experimento, tem-se como medidas diretas, o raio do feixe de elétrons formado dentro do tubo, sendo utilizada uma régua que a menor partição tem $1mm$ o erro associado é de $0,5mm$ ou $0,0005m$, e também a amperagem e voltagem utilizada nas bobinas de Helmholtz, sendo um multímetro digital, seu erro é $\pm 1$ a menor unidade.
  \subsubsection{Desvio de Medidas Indiretas}
  Para o cálculo da razão carga massa é utilizada a equação \ref{e/m}, mas substituindo $V$ e $r^2$ por $\lambda$ se obtém a igualdade \ref{eq1} e para o cálculo dos erros, é aplicado o logaritmo neperiano em ambos os lados da equação, tornando-se:
  \begin{equation}
      ln\left(\frac{e}{m}\right)=ln\left(\frac{2\lambda\left(\frac{5}{4}\right)^3a^2}{[N\mu _0 I]^2}\right)
  \end{equation}
   utilizando propriedades de logaritmos, é obtido
  \begin{equation}
      ln(e/m)=ln\left(2\lambda\left(\frac{5}{4}\right)^3a^2\right)-ln\left([N\mu _0 I]^2\right)
  \end{equation}
  \begin{equation}
      ln(e/m)=ln(2\lambda)+3ln(5/4)+2ln(a)-2(ln(N)+ln(\mu _0)+ln(I))
  \end{equation}
  diferenciando a equação e eliminando os termos constantes ou sem erro associado
  \begin{equation}
      \frac{de/m}{e/m}=\frac{d\lambda}{\lambda} + \frac{2dI}{I}
  \end{equation}
  e fazendo $d\rightarrow \delta$ para poder ser aplicado os erros medidos experimentalmente. Onde $\delta$ representa o erro associado.
  \begin{equation}
      \frac{\delta e/m}{e/m}=\frac{\delta\lambda}{\lambda} + \frac{2\delta I}{I}
  \label{eq:err}
  \end{equation}

