\section{Resumo}
O presente trabalho tem como objetivo determinar a razão carga-massa do elétron por meio de um experimento feito com materiais da Pasco. O relatório aborda conceito histórico do experimento utilizado descrevendo os materiais, o método de montagem, execução do mesmo e dados obtidos, fundamentação teórica e dedução das equações utilizadas, analise dos dados e resultados obtidos, e conclusões.
\section{Introdução}
No fim do século XIX surgiram as primeiras estimativas da grandeza da razão e/m com o físico Pieter Zeeman, inspirado pelos trabalhos de Faraday o neerlandês obteve provas de partículas atômicas com relação definida entre carga e massa. Zeeman, por meio de uma análise espectral da luz emitida pelos átomos na presença de um campo magnético, concluiu que as partículas que emitiam luz nos átomos possuíam carga negativa, também encontrou um valor para e/m de $1,6\times10^{11} C/kg$, valor muito próximo ao das medições atuais ($1,7596\times10^{11}C/kg$).

mson, com um experimento muito parecido com o que foi reproduzido para este relatório, obteve por meio de medidas diretas valores para a razão carga-massa, utilizando um tubo de raios catódicos, uma voltagem especifica, e um campo magnético perpendicular ao campo elétrico gerado pela tenção aplicada. Thomson media a deflexão que o campo magnético gerava, por meio deste método o cientista não obteve um valor muito satisfatório ($0.7\times10^{11} C/kg$) – apesar de já ter obtido valores mais próximos ao real em diferentes experimentos com a mesma finalidade - devido ao fato de não ter levado em consideração o campo magnético da terra que podia influenciar o raio magnético experimental após a interação com o campo criado por ele. 
