\section{Conclusão}
Proposto em 1851, o experimento de Fizeau apresenta uma alta precisão para seu tempo, faltando apenas algumas refinações (como as propostas por Foucault e Michelson-Morley) para estimar a velocidade da luz. O experimento realizado em laboratório permitiu definir a velocidade da luz com uma alta precisão em relação ao valor aceito atualmente. Os valores obtidos não são exatos pois o experimento foi realizado num ambiente rodeado por ar, não em vácuo, e ainda vale atribuir as divergências de valores ao erro experimental das medidas e ao movimento das pessoas no laboratório.
 
