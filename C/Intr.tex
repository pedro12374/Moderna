\section{Introdução}
Desde os primórdios da ciência o homem se preocupa em estudar a luz. Os
filósofos atomistas da Grécia Antiga já tratavam a luz como composto
por pequenas partículas chamadas de "corpúsculos", que descreveriam
uma trajetória em linha reta com velocidade limitada. Tal ideia foi
adotado pelo físico inglês Isaac Newton que fez seus estudos de ótica
baseando-se nessa teoria. Foi Christiann Huygens quem desenvolveu a
teoria ondulatório da luz, ao comprovar que esta se comporta como uma
onda se propagando no éter. No fina do século XIX, a luz era concebida
como uma onda eletromagnética, mas tal concepção passa a ser
questionada quando, ao incidir com algum material, a luz se comportava
como partícula, não como onda. Baseado nas ideias de Max Planck, o
físico Albert Einstein demonstrou através da comprovação do efeito
fotoelétrico que a luz, ao incidir sobre um material, age como
pequenos pacotes de energia, os chamados fótons. Quando Arthur Compton
comprovou que um fóton ao colidir com um elétron se comporta como
uma partícula material, definiu-se a dualidade onda-partícula da luz -
comporta-se como onda no vácuo, mas como partícula ao incidir numa
superfície.

Tais estudos e avanços levaram à pergunta "qual a velocidade da luz?".
Galileu Galilei tentou estimar esse valor com o seguinte experimento:
um observador A e um observador B estão posicionados a alguns
quilomêtros de distância, tanto A quanto B carregam consigo lanternas.
Galileu supôs que quando o obervador A apagasse sua luz, o obversador
B veria este evento e também apagaria a sua luz, o observador A
mediria então esse intervalo de tempo e seria capaz de determinar a
velocidade da luz. Galileu chegou ao resultado de que a luz tem
"velocidade infinita".

Num experimento mais complexo, o físico Fizeau mediu a velocidae da
luz ao incidar um feixe luminoso sobre uma engrenagem giratória
espelhada que refletiria a luz para um espelho fixo, desse espelho a
luz voltaria para a engrenagem. Através da variação de posição entre a
luz infringida e a refletida pela engranagem, Fizeau foi capaz de
estimar uma valor aporoximado para a velocidade da luz.

Numa tentativa de aperfeiçoar o experimento de Fizeau, Foucault o
realizou substituindo a engrenagem por um espelho rotatório. Esse
relatório apresentará uma reprodução do experimento realizado por
Foucault e, posteriormente, Michelson e Morley ao estudar a difração
da luz.
