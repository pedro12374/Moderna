\section{Introdução}
Interferência é um fenômeno que ocorre quando duas ou mais ondas sobrepostas se encontram fora de fase, gerando interferências construtivas e destrutivas conforme a interação entre as ondas, sendo a luz uma onda eletro magnética o fenômeno também se aplica a ela, como foi observado no experimento de fenda dupla de Young, as ondas de luz emitidas a partir das fendas se sobrepõem e criam faixas escuras e claras sendo estas os mínimos e máximos de intensidade luminosa gerados pela interferência. Michelson montou o seu interferômetro, aparato que foi utilizado no experimento contido neste relatório, a fim de mostrar a influência do "éter luminoso"(suposto meio necessário para a propagação da luz) sobre a velocidade da luz, o experimento mostrou a não existência do éter. 
No presente relatório, o interferômetro de Michelson será utilizado para determinar o comprimento de onda de um lazer.
